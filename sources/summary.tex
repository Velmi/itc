\section{Einleitung}

\subsection{Informationstheorie}

Bit: binary unit $\rightarrow$ Einheit für Information

bit: binary digit $\rightarrow$ bit als binäres Symbol

\textbf{Informationgehalt} 

je unwahrscheinlicher ein Symbol $x$ auftritt, desto mehr Information enhält es:

$\displaystyle{
    I(x) = ld\left( \frac{1}{P(x)} \right) = -ld(P(x))
}$

$P$: Wahrscheinlichkeit eines Symbols\\
$I$: Informationsgehalt $[I] = Bit$

\textbf{Entropie}

gemittelter Informationsgehalt einer Quelle $X$:

$\displaystyle{
    H(X) = \sum_{i} P(x_i) \cdot I(x_i) = - \sum_{i} P(x_i) \cdot ld(P(x_i))
}$

$H$: Entropie $[H] = Bit/Symbol$

\textbf{Entscheidungsgehalt}

Entropie wird maximal, wenn alle Symbole gleichwahrscheinlich sind
$\rightarrow$ Entscheidungsgehalt

$\displaystyle{
    H_0 = ld(N)
}$

$H_0$: Entscheidungsgehalt $[H_0] = Bit/Symbol$\\
$N$: Anzahl der Symbole eines Alphabets

\textbf{Redundanz}

$\displaystyle{
    R = H_0 - H
}$

$\displaystyle{
    r = \frac{R}{H_0}
}$

$R$: Redundanz $[R] = Bit/Symbol$\\
$r$: relative Redundanz

%%//TODO: mittlere Laenge Code

\subsection{Quellcodierung}

\subsubsection{Huffman-Code}

ist \textbf{Präfixcode:} ein Codewort ist niemals Anfang eines anderen Codewortes

Codebaum aufbauen:
\begin{enumerate}
    \item Ordne die Symbole nach Auftrittswahrscheinlichkeit
    \item Fasse Symbole mit niedrigster Wahrscheinlichkeit zu einem Symbol zusammen und addiere die Wahrscheinlichkeiten
    \item Wiederhole bis nur ein Symbol übrig bleibt
\end{enumerate}

Beschrifte die Pfade mit 1 und 0\\
$\rightarrow$ Codewort ergibt sich, indem man von Wurzel bis zum Blatt geht

\includegraphics[width=6cm]{img/huffman.PNG}

\underline{Hinweis}: Beschriftung der 0; 1 theoretisch egal aber für Überprüfung mit Onlinerechnern
sollte konsistent der Pfad mit der geringeren und höheren Wahrscheinlichkeit gleich beschriftet
werden

\subsubsection{Arithmetische Codierung}

Codierung eines Wortes (oder Textes) durch Zahl

Endezeichen notwendig, da keine natürliche Terminierung des Codes

%//TODO arithmetische Codierung

\subsection{Kanalmodell}

\subsubsection{Binärkanal}

\includegraphics[width=4cm]{img/kanalmodell.PNG}

$P(Y|X)$: \frqq Wahrscheinlichkeit für Y, wenn X gesendet wurde\flqq (a-priori-Wahrscheinlichkeit)

$P(X|Y)$: \frqq Wahrscheinlichkeit für X, wenn Y empfangen wurde\flqq (a-posteriori-Wahrscheinlichkeit)

\textbf{Unxymmetrisch}: Fehlerwahrscheinlichkeit für \frqq 0\flqq{} und \frqq 1\flqq{} unterschiedlich

$\displaystyle{
    \begin{pmatrix}
        P(Y|X)    
    \end{pmatrix}
    =
    \begin{pmatrix}
        P(0|0) & P(0|1)\\
        P(1|0) & P(1|1)
    \end{pmatrix}
}$

\textbf{Symmetrisch}: Bitfehlerwahrscheinlichkeit $P_e$ für \frqq 0\flqq{} und \frqq 1\flqq{} gleich

$\displaystyle{
    \begin{pmatrix}
        P(Y|X)    
    \end{pmatrix}
    =
    \begin{pmatrix}
        P(0|0) & P(0|1)\\
        P(1|0) & P(1|1)
    \end{pmatrix}
    =
    \begin{pmatrix}
        1-P_e & P_e\\
        P_e & 1-P_e
    \end{pmatrix}
}$

\subsubsection{Bedingte Entropie}

$\displaystyle{
    H(Y|X) = - \sum_{j} \sum_{i} P(x_i, y_i)\,ld(P(y_i|x_i))
}$\;\; \textbf{Irrelevanz}

$\displaystyle{
    H(X|Y) = - \sum_{j} \sum_{i} P(x_i, y_i)\,ld(P(x_i|y_i))
}$\;\; \textbf{Äquivokation}

$H(Y|X)$: \textbf{Irrelevanz (Fehlinformation)}, mittlere Unsicherheit über empfangene Symbole, wenn Sendesymbole bekannt\\
$H(X|Y)$: \textbf{Äquivokation (Informationsverlust)}, mittlere Unsicherheit über gesendete Symbole, wenn Empfangssymbole bekannt

\subsubsection{Entropiemodell}

\includegraphics[width=7cm]{img/entropiemodell.PNG}

für Empfänger ist $H(Y)$ (also der Ausgang des Kanals) eine neue Quelle, allerdings mit zusätzlicher Fehlinformation

Die Transinformation ist dann die Entropie des Kanals minus der Fehlinformation

$\displaystyle{
    T(X;Y) = H(Y) - \underbrace{H(Y|X)}_{\text{Fehlinformation}}
}$

oder die Entropie der Quelle minus des Informationsverlusts

$\displaystyle{
    T(X;Y) = H(X) - \underbrace{H(X|Y)}_{\text{Informationsverlust}}
}$

Fehlinformation größer als Informationsverlust (außer es passieren keine Fehler,
oder die Entropie der Quelle ist maximal, dann $H(X|Y) = H(Y|X)$)

$\hookrightarrow$ Entropie am Ausgang des Kanals größer als die der Quelle (außer es passieren keine Fehler, oder die
Entropie der Quelle ist maximal, dann $H(X) = H(Y) = 1$)

\subsection{Kanalkapazität}

Die Kanalkapazität ist das Maximum an Information, welche über den Kanal geht

$\displaystyle{
    C = \max(T(X;Y)) = \max(H(X) - H(X|Y))
}$

$\displaystyle{
\;\;\;= \max(H(Y) - H(Y|X))
}$

$\displaystyle{
\;\;\;= 1 - H(X|Y) = 1 - H(Y|X)
}$

für binären, symmetrischen Kanal:

$\displaystyle{
    H(Y|X) = -p \cdot ld(p) - (1 - p) \cdot ld(1 - p)
}$

$\displaystyle{
    C = 1 + p \cdot ld(p) + (1 - p) \cdot ld(1 - p)
}$

$\displaystyle{
    C_S = C \cdot R_S
}$

$p$: Bitfehlerwahrscheinlichkeit\\
$C$: Kanalkapazität in Information/Symbol $[C] = Bit/Symbol$\\
$C_S$: Kanalkapazität in Information/Zeit $[C_S] = Bit/s$\\
$R_S$: Symbolrate $[R_S] = Symbol/s$

\textbf{Informationsfluss}

$\displaystyle{
    H_S = H \cdot R_S
}$

$H_S$: Informationsfluss in Information/Zeit $[H_S] = Bit/s$\\
$H$: Informationsgehalt pro Symbol $[H] = Bit/Symbol$\\
$R_S$: Symbolrate $[R_S] = Symbol/s$

\subsection{Shannon-Theoreme}

\subsubsection{Erster Satz von Shannon}

Quellcodierung

Ist ein Informationsfluss von $H_S$ gewünscht, welcher kleiner als die Kanalkapazität $C_S$ ist,
ist es möglich einen Code zu finden, sodass der Informationsfluss übertragen werden kann

\subsubsection{Zweiter Satz von Shannon}

Kanalcodierung

Ist der die gewünschte Datenfluss $R_S$ ($[R_S] = Symbol/s = bit/s$) kleiner als die Kanalkapazität $C_S$, dann
gibt es einen Code, sodass die Fehlerwahrscheinlichkeit beliebig klein wird

\subsubsection{Shannon-Hartley}

Kanalkapazität abhängig vom SNR

$\displaystyle{
    C_S = W \cdot ld\left( 1 + \frac{S}{N} \right)
}$

$C_S$: Kanalkapazität in Information/Zeit $[C_S] = Bit/s$\\
$W$: Bandbreite $[W] = Hz$\\
$S$: Signalleistung\\
$N$: Rauschleistung

\textbf{Bandbreiteneffizienz}

Wird mit $R_S = C_S$ übertragen, kann die Bandbreiteneffizienz bestimmt werden

$\displaystyle{
    \frac{R_S}{W} = ld\left( 1 + \frac{S}{N} \right)
}$

$\displaystyle{
    S = \frac{E_b}{T_b} = E_b \cdot R_S
}$\;\;\;\;\;\;\;\;\;\;
$\displaystyle{
    N = N_0 \cdot W
}$

$\displaystyle{
    \frac{S}{N} = \frac{E_b \cdot R_S}{N_0 \cdot W}
}$

$\hookrightarrow$ damit S/N \underline{und} Bandbreiteneffizienz abhängig von $W$

Ist eine Bandbreiteneffizienz gewünscht, kann das erfolderliche S/N ausgerechnet werden

$\displaystyle{
    \frac{E_b}{N_0} = \frac{W}{R_S} \left( 2^{\frac{R_S}{W}} - 1 \right)
}$

$\displaystyle{
    \frac{E_b}{N_0}_{min} = -1,6\,dB
}$

$\frac{R_S}{W}$: Bandbreiteneffizienz $[\frac{R_S}{W}] = \left(\frac{bit/s}{Hz}\right)$\\
$E_b$: Energie die für die Übertragung von 1 bit aufgewendet wird $[E_b] = J = Ws = W/Hz$\\
$T_b$: Zeitdauer eines bits $[T_b] = s = 1/Hz$

\section{Blockcodes}

Code beschrieben durch $C(n, k, d)$

$n$: Länge Codewort\\
$k$: Länge Informationswort\\
$d$: Mindestabstand

Coderate: 
$\displaystyle{
    CR = \frac{k}{n}
}$

\subsection{Generelles}

\textbf{Generatormatrix}

Erzeugung eines Codewortes über Multiplikation eines Informationsvektors

$\displaystyle{
    \vec{c} = i \cdot G
}$

$c$: Codewort\\
$i$: Informationswort\\
$G$: Generatormatrix

\textbf{Prüfmatrix}

$\displaystyle{
    H \vec{c}^T = 0
}$

$\displaystyle{
    H \cdot G^T
}$

$H$: Prüfmatrix

\textbf{Syndrom}

wenn Empfangswort $r$ fehlerhaft ist (und nicht zum Coderaum gehört) dann ist das Produkt aus
Prüfmatrix und Empfangswort das \textit{Syndrom} und nicht mehr 0

$\displaystyle{
    H \vec{r}^T = H (\vec{c} + \vec{f})^T = H \vec{f}^T = \vec{s} \neq 0
}$

\textbf{Systematische Codes}

Systematischer Code: Einheitsmatrix $I$ ist Teil der Generatormatrix

$\displaystyle{
    G = [I | G']
}$

Bsp.: C(7, 4, 3): 
$\displaystyle{
    G =
    \begin{pmatrix}
        1 & 0 & 0 & 0 & 0 & 1 & 1\\
        0 & 1 & 0 & 0 & 1 & 0 & 1\\
        0 & 0 & 1 & 0 & 1 & 1 & 0\\
        0 & 0 & 0 & 1 & 1 & 1 & 1
    \end{pmatrix}
}$

\textbf{Lineare Codes}

Eigenschaften:
\begin{itemize}
    \item Mindestgewicht = Mindestabstand
    \item Codewort + anderes Codewort = wieder Codewort
    \item Nullwort ist teil des Codes
\end{itemize}

\textbf{Gewicht}

Gewicht eines Codewortes: Anzahl der von 0 verschiedenen Stellen

Mindestgewicht: Minimale Anzahl an Stellen, die von 0 verschieden sind

\textbf{Distanz/Abstand}

Hamming-Distanz: Anzahl verschiedener Stellen zweier Codewörter

Mindestdistanz: Mindestanzahl verschiedener Stellen zweier beliebiger Codewörter eines Codes

\textbf{Fehlerkorrektur/Fehlererkennung}

Anzahl der korrigierbaren Fehler $e$ bei gegebenem Abstand $d$

$\displaystyle{
    e = \Bigl\lfloor \frac{d - 1}{2} \Bigr\rfloor
}$

\underline{bei ungeradem $d$:}

es werden $d-1$ Fehler erkannt

\underline{bei geradem $d$:}

es werden alle ungeraden Anzahlen an Fehlern erkannt

\subsection{Fehlerwahrscheinlichkeit}

\textbf{Restblockfehlerwahrscheinlichkeit}

Wahrscheinlichkeit, dass von $n$ Symbolen beliebige $m$ falsch und die übrigen Symbole $n-m$ richtig sind:

$\displaystyle{
    \binom{n}{m} \cdot p^m \cdot (1 - p)^{n-m}
}$

\underline{für ungerades $d$:}

alle Fehler werden entweder korrigiert, oder nicht erkannt

Kann ein Code $e$ Fehler korrigieren, dann verbleibt eine Restblockfehlerwahrscheinlichkeit von:

$\displaystyle{
    P_{Block} = \sum_{m = e + 1}^{n} \binom{n}{m} \cdot p^m \cdot (1 - p)^{n-m}
}$

$\displaystyle{
    = 1 - \sum_{m=0}^{e} \binom{n}{m} \cdot p^m \cdot (1 - p)^{n-m}
}$

$p$: Fehlerwahrscheinlichkeit vor der Decodierung\\
$P_{Block}$: Wahrscheinlichkeit, für Symbolfehler im decodierten Block

\underline{für gerades $d$:}

alle ungeraden Fehler werden erkannt, da sie nicht in Korrekturkugeln liegen; können damit aber auch
nicht korrigiert werden

$\rightarrow$ für die Wahrscheinlichkeit, dass Fehler nicht erkannt werden, tragen also nur die geraden Fehleranzahlen, welche größer als $e$ sind, bei

$\displaystyle{
    P_{Block,erkennen} = \sum_{m=e+1}^{n} \binom{n}{2m} \cdot p^{2m} \cdot (1 - p)^{n-2m}
}$

$P_{Block}$ gleich wie bei gerader Anzahl

\textbf{Symbolfehlerwahrscheinlichkeit}

Block besteht aus $k$ Symbolen

Bei gegebener Restblockfehlerwahrscheinlichkeit, gibt es eine Symbolfehlerwahrscheinlichkeit von

$\displaystyle{
    P_{Symb} = P_{S|Block} \cdot P_{Block} = \frac{2^{k-1}}{2^k - 1} P_{Block}
}$

\subsection{Hamming-Codes}

immer $d=3$, damit immer $e=1$

$\displaystyle{
    n = 2^h - 1
}$

$\displaystyle{
    k = n - h
}$

\begin{tabular}{c|c c c}
    h & n & k & d\\
    \hline
    2 & 3 & 1 & 3\\
    3 & 7 & 4 & 3\\
    4 & 15 & 11 & 3\\
    5 & 31 & 26 & 3\\
    \vdots & \vdots & \vdots & \vdots
\end{tabular}

\textbf{Konstruktion}

\underline{1. Erstellung Prüfmatrix}

Spalten sind Dualdarstellung der Spaltennummer

$H_{n-k x n}$

Beispiel: C(7, 4, 3)

$\displaystyle{
    H_{3 x 7} = \begin{pmatrix}
        0 & 0 & 0 & 1 & 1 & 1 & 1\\
        0 & 1 & 1 & 0 & 0 & 1 & 1\\
        1 & 0 & 1 & 0 & 1 & 0 & 1
    \end{pmatrix}
}$

\underline{2. Spalten tauschen, sodass hinten Einheitsmatrix}

$H = [A_{n-k x k} | I_{n-k x n-k} ]$

Spalte 1 mit 7\\
2 mit 6\\
4 mit 5

$\displaystyle{
    \rightarrow H = \begin{pmatrix}
        1 & 1 & 0 & 1 & 1 & 0 & 0\\
        1 & 1 & 1 & 0 & 0 & 1 & 0\\
        1 & 0 & 1 & 1 & 0 & 0 & 1
    \end{pmatrix} = [ A_{3 x 4} | I_{3 x 3} ]
}$

\underline{3. Generatormatrix aufstellen}

$G_{k x n} = [I_{k x k} | -A^T_{k x n-k} ]$

$\displaystyle{
    G = \begin{pmatrix}
        1 & 0 & 0 & 0 & 1 & 1 & 1\\
        0 & 1 & 0 & 0 & 1 & 1 & 0\\
        0 & 0 & 1 & 0 & 0 & 1 & 1\\
        0 & 0 & 0 & 1 & 1 & 0 & 1
    \end{pmatrix}
}$

systematisch, da Informationswort zusammenhängend im Codewort steht

\underline{4. Rücktauschen der Spalten}

Spalte 1 mit 7\\
2 mit 6\\
4 mit 5

$\displaystyle{
    G = \begin{pmatrix}
        1 & 1 & 0 & 1 & 0 & 0 & 1\\
        0 & 1 & 0 & 1 & 0 & 1 & 0\\
        1 & 1 & 1 & 0 & 0 & 0 & 0\\
        1 & 0 & 0 & 1 & 1 & 0 & 0
    \end{pmatrix}
}$

quasi-systematisch, da Informationswort zwar im Codewort, aber nicht zusammenhängend

\textbf{Fehler und Syndrom}

Fehler gibt an, an welcher Stelle des Empfangsworts ein Fehler aufgetreten ist

Bsp.:

$\displaystyle{
    \vec{s} = \begin{pmatrix}
        0\\
        1\\
        1
    \end{pmatrix}
}$

$\rightarrow$ Fehler an Stelle $(011)_b = 3$ im Empfangswort

$\rightarrow$ da binärer Code, muss für Fehlerkorrektur das dritte bit nur invertiert werden

\section{Galois-Felder}
\label{sec:galois}

\subsection{Algebraische Strukturen}

\includegraphics[width=8cm]{img/algebraische_strukturen.PNG}

\textbf{Menge}

Verbund von Elementen, welche keine Operationen beinhalten (Möbel können eine Menge sein, es kann aber nicht Tisch + Stuhl gerechnet werden)

\textbf{Halbgruppe}

Menge $A$ mit Verknüpfung \frqq +\flqq{} ist eine Halbgruppe, wenn
\begin{itemize}
    \item Abgeschlossenheit (+ zweier Elemente von $A$ ergibt wieder ein Element von $A$)
    \item Assoziativität (Reihenfolge der Operation mit + spielt keine Rolle, $a + (b + c) = (a + b) + c$)
    \item Existenz eines neutralen Elements (Element $a$ + neutrales Element $n$ ergibt wieder Element $a$)
\end{itemize}

\textbf{Gruppe}

Halbgruppe plus
\begin{itemize}
    \item Existenz eines additiven inversen Elements ($a + b = n$)
\end{itemize}

\textbf{Abelsche oder kommutative Gruppe}

Gruppe plus
\begin{itemize}
    \item Kommutativität (Reihenfolge der Operanden spielt keine Rolle, $a + b = b + a$)
\end{itemize}

\textbf{Ring}

abelsche Gruppe plus
\begin{itemize}
    \item Abgeschlossenheit bezüglich \frqq $\cdot$\flqq
    \item Assoziativität bezüglich \frqq $\cdot$\flqq
    \item Distributivität ($a \cdot (b + c) = a \cdot b + a \cdot c$)
\end{itemize}

\textbf{Körper}

Ring plus
\begin{itemize}
    \item Kommutativität bezüglich \frqq $\cdot$\flqq ($a \cdot b = b \cdot a$)
    \item Neutrales Element bezüglich \frqq $\cdot$\flqq
    \item Inverses Element bezüglich \frqq $\cdot$\flqq für jedes Element
\end{itemize}

\textbf{Primkörper/Galois-Feld}

Körper, indem Addition und Multiplikation $\mod p$ gerechnet wird ($p$ muss dabei eine Primzahl sein)\\
$\hookrightarrow GF(p)$

\subsection{Eigenschaften Galois-Felder}

\textbf{Primitives Element}

Element $\alpha$, welches durch ihre $p-1$ Potenzen alle Elemente (außer $0$) des $GF(p)$ erzeugt

Bsp. $GF(5), \alpha = 2$:

$2^0 = 1 \mod 5 = 1$\\
$2^1 = 2 \mod 5 = 2$\\
$2^2 = 4 \mod 5 = 4$\\
$ 2^3 = 8 \mod 5 = 3$

ab hier zykische Wiederholung:\\
$2^4 = 16 \mod 5 = 1$

\textbf{Polynome}

Folge an $n$ Zahlen im Galois-Feld wird als Polynom vom Grad $n-1$ geschrieben

$\displaystyle{
    \hookrightarrow \{ 1; 4; 3; 1 \} \rightarrow A(x) = 1x^3 + 4x^2 + 3x + 1
}$

Auswertung des Polynoms $A(x)$ an verschiedenen Stellen von $\alpha^i$ ergibt ihre Fouriertransformierte $a(x)$

$\displaystyle{
    a_i = A(\alpha^i)
}$

\textbf{Zyklische Faltung}

Polynommultiplikation im Galois-Feld $\rightarrow$ zyklische Faltung

Normale Faltung mit endlichen Signalen $\rightarrow$ endliches Faltungsergebnis

Zyklische Faltung: Signale sind periodisch, damit Faltungsergebnis ebenfalls periodisch (und damit unendlich lang)

\includegraphics[width=8.7cm]{img/zykische_faltung.PNG}

links: zyklische Faltung \;\;\;\;\;\;\;\;\;\;\;\; rechts: normale Faltung

\section{Reed-Solomon-Code}

\subsection{Wunsch und Idee}

\textbf{Wunsch}

Konstruktion eines Codes mit vorgegebener Korrekturfähigkeit\\
$\rightarrow$ Vorgabe des Mindestabstandes $d$

$\displaystyle{
    e = \Bigl\lfloor \frac{d - 1}{2} \Bigr\rfloor
}$\\
$\displaystyle{
    d = 2e + 1
}$

bei linearem Code ist Mindestabstand = Mindestgewicht

$\rightarrow$ Codeworte haben mind. $d$ von 0 verschiedene Koeffizienten

d'Alembert: Polynom vom Grad $n$ hat $n$ komplexe (oder höchstens $n$ reelle) Nullstellen; auch
im Galois-Feld

\textbf{Idee}

Konstruktion des Informationswortes als Polynom $A(x)$ mit Grad $k-1$ (damit höchstens $k-1$ Nullstellen)

Im $GF(p)$ mit Ordnung $n = p-1$ kann man $A(x)$ an $n$ Stellen auswerten, danach wiederholen sich die Werte

$\rightarrow$ Auswertung des Polynoms für verschiedene $x$ (bzw. $\alpha^i$) ergeben die Koeffizienten $a_i$ des
Polynoms $a(x)$

$\displaystyle{
    a_i = A(\alpha^i) \qquad\qquad\qquad \text{\textbf{IDFT}}
}$

von diesen sind höchstens $k-1$ Null (weil $grad(A(x)) = k-1$)\\
von diesen sind also mind. $n - (k-1)$ von Null verschieden $\rightarrow$ Mindestgewicht $d$

$d = n - (k - 1) = n - k + 1$

Rücktransformation des Codeworts in Informationswort:

$\displaystyle{
    A_i = n^{-1} a(\alpha^{-i}) \qquad\qquad \text{\textbf{DFT}}
}$

\subsection{Codierung}

Verschiedene Möglichkeiten aus einem Informationswort ein Codewort zu generieren

\subsubsection{Generatorpolynom}
\label{subsubsec:gen-poly}

Erzeugt zusammenhängende Nullstellen im Codewort = Syndromstellen

$\displaystyle{
    g(x) = \prod_{i=k}^{n-1} \left(x - \alpha^{-i}\right)
}$

Syndromstellen beginnen hier bei $k$, es sind aber alle anderen Stellen möglich, solange sie zusammenhängen

$grad(g(x)) = d-1 = n-k$ = Anzahl Syndromstellen

$g(x)$: Generatorpolynom\\
$i(x)$: Informationspolynom

\subsubsection{Prüfpolynom}

Prüfpolynom:

$\displaystyle{
    h(x) = \prod_{i=0}^{k-1} \left(x - \alpha^{-i}\right)
}$

Produkt aus Generator- und Prüfpolynom ist 0

$\displaystyle{
    g(x) \cdot h(x) = 0
}$

und Produkt aus Codepolynom und Prüfpolynom ist 0

$\displaystyle{
    a(x) \cdot h(x) = 0
}$

genau da, wo $g(x)$ (oder $a(x)$) Nullstellen hat (also $G_i$ 0 ist) hat das Prüfpolynom $h(x)$ keine Nullstellen
(ist also $H_i$ nicht 0) und umgekehrt

\subsubsection{IDFT (nicht systematisch)}

$\displaystyle{
    a_i = A(\alpha^i)
}$

$A(x)$: Informationswort\\
$a_i$: Koeff. des Codewortes

\subsubsection{Polynommultiplikation (nicht systematisch)}

$\displaystyle{
    a_i = g(x) \cdot i(x)
}$

\subsubsection{Polynomdivision (systematisch)}

Informationswort ist Teil des Codewortes (an den hohen Potenzen)

$\displaystyle{
    a^*(x) = i_{k-1} x^{n-1} + i_{k-2} x^{n-2} + ... + i_1 x^{n-k+1} + i_0 x^{n-k}
}$

jedes Codewort muss durch Generatorpolynom teilbar sein $\rightarrow$ ist
für $a^*(x)$ i.A. nicht der Fall

$\displaystyle{
    \frac{a^*(x)}{g(x)} = b(x) + \frac{rest(a^*(x))}{g(x)}
}$\\
$\displaystyle{
    \rightarrow \frac{a^*(x) - rest\left(a^*(x)\right)}{g(x)} = b(x)
}$\\
$\displaystyle{
    a(x) = a^*(x) - rest\left(a^*(x)\right)
}$

$rest\left(a^*(x)\right)$: Divisionsrest

\subsubsection{Zyklischer Code}

Multiplikation eines Polynoms mit $x^i$ verschiebt Koeff. des Polynoms um $i$-Stellen

durch mod-Rechnung des Exponenten verschieben sich höhere Exponenten wieder an den Anfang des Polynoms

Bsp.:

$\displaystyle{
    x \cdot a(x) = x \cdot ( 2x^2 + x + 1 ) = 2x^3 + x^2 + x = x^2 + x + 2
}$

\subsection{Decodierung}

\textbf{Idee:}

Addition des Fehlerpolynoms $f(x)$ mit $t$ Koeffizienten (d.h. $t$ Fehler sind auf dem Kanal aufgetreten)
zum gesendeten Codewort $a(x)$

im Zeitbereich:

$\displaystyle{
    r(x) = a(x) + f(x)
}$

im Frequenzbereich:

$\displaystyle{
    R(x) = A(x) + F(x)
}$

gedanklich wird ein Polynom $c(x)$ aufgestellt, welches $t$ Nullen an den Fehlerstellen hat

Da die Koeffizienten von $c(x)$ die Auswertung ihrer Fouriertransformierten $C(x)$ ist, ist der Grad
von $C(x)$ $t$

Da $c(x)$ gerade dort 0 ist, wo $f(x)$ ungleich 0, ist das Produkt $f_i \cdot c_i$ immer 0 (Achtung, keine
Polynommultiplikation gemeint, sondern punktweise Multiplikation)

$\displaystyle{
    f_i \cdot c_i = 0
}$

wenn Zeitbereich = 0 $\rightarrow$ Frequenzbereich = 0

$\displaystyle{
    F(x) \cdot C(x) = 0
}$

Achtung: hier Polynommultiplikation/ Faltung/ Filterung gemeint\\
$\hookrightarrow$ Aufstellen der Schlüsselgleichungen

\includegraphics[width=8cm]{img/decod_rs.PNG}

\subsubsection{Vorgehen}

\begin{enumerate}
    \item Fouriertransformation des empfangenen Codewortes $r(x) \rightarrow R(x)$
    \item Auslesen der Koeff. des Syndrompolynoms ($S_0, ..., S_n$) aus $R(x)$ und Aufstellen des Syndrompolynoms
    \item Berechnung des $C(x)$ aus Schlüsselgleichungen oder euklidschem Algorithmus
    \item Berechnung der Fehlerstellen durch Nullstellensuche von $C(x)$
    \item Berechnung des Fehlerwertes über Schlüsselgleichungen oder Forney-Algorithmus
\end{enumerate}

\subsubsection{Schlüsselgleichungen}

beschreiben, dass Faltung von $C(x)$ und $F(x)$ Null ist (Achtung: zyklische Faltung, siehe \autoref{sec:galois})

$F_0$ bis $F_{n-k-1}$ (bzw. $F_{d-2}$) sind bekannt, da diese direkt an den Syndromstellen
von $R(x)$ stehen

Alle $C$-Koeff. sind unbekannt, außer $C_{t}$, dieser wird zu $1$ gesetzt

$\displaystyle{
    C_{t} = 1
}$

da Anzahl der Fehler ($t$) unbekannt ist, muss ausprobiert werden, welche \underline{minimale} Anzahl an Fehlern
die Schlüsselgleichungen widerspruchsfrei erfüllt

Lösen der Schlüsselgleichungen nach $C(x)$

$\hookrightarrow$ Nullstellensuche von $C(x)$ ergibt die Nullen des $c(x)$

$\hookrightarrow$ wenn Grad von $C(x)$ nicht mit Anzahl der Nullstellen übereinstimmt $\rightarrow$ Decodierversagen

Lösen der Schlüsselgleichungen nach $F(x)$

$\hookrightarrow f(x)$ aus Rücktransformation von $F(x)$

$\hookrightarrow f(x)$ von $r(x)$ abziehen, man erhält $a(x)$

$\displaystyle{
    a(x) = r(x) - f(x)
}$

%%// TODO: Schlüsselgleichungen aufstellen und eigentliche Schlüsselgleichungen

\subsubsection{Euklidscher Algorithmus}

Suche des ggT zweier Zahlen

Kann zur Lösung der Schlüsselgleichungen verwendet werden

Rest:

$\displaystyle{
    r_n = v_n a_n + w_n b_n
}$

Rekursionsformeln für $v_n$ und $w_n$:

$\displaystyle{
    v_n = v_{n-2} - q_n v_{n-1}
}$

$\displaystyle{
    w_n = w_{n-2} - q_n w_{n-1}
}$

$q_n$: Quotient des vorherigen Schrittes

Initialisierung:

$\displaystyle{
    v_{-1} = 1 \;\;\;\;\; v_0 = 0
}$\\
$\displaystyle{
    w_{-1} = 0 \;\;\;\;\; w_0 = 1
}$

Suche des $C(x)$ und damit den Fehlerstellen

Polynomdivision von $x^{d-1}$ und des Syndrompolynoms $S(x)$

$\displaystyle{
    x^{d-1} : S(x)
}$

Wenn Rest der Division im Grad nicht kleiner ist als die Anzahl der Fehler $e$, die maximal korrigiert
werden können $\rightarrow$ weiter: $S(x) : r_1(x)$

usw.

ist Grad des Restes kleiner als $e$ $\rightarrow$ Berechnung des $C(x)$ und des $T(x)$

$\hookrightarrow C(x) = w_n$

$\hookrightarrow T(x) = -r_n$

\subsubsection{Forney-Algorithmus}

Fehlerstellenberechnung durch Nullstellensuche in $C(x)$

$\displaystyle{
    c_i = C(\alpha^i)
}$

Fehlerwertberechnung aus gegebenem $C(x)$ und $T(x)$

$\displaystyle{
    f_i = x^q \cdot n \cdot x^{-1} \frac{T(x)}{C'(x)}\bigg \vert_{x=\alpha^i}
}$

$q$: Verschiebung der Syndromstellen ($q=5$, wenn Syndrom an Stelle 5)

\underline{1. Hinweis}: Fehlerwert an den Stellen, an dem \underline{keine} Fehler passiert sind, ist im Allgemeinen
\underline{nicht} 0

\underline{2. Hinweis}: Ableitungsregeln beachten, Exponent der beim Ableiten als Faktor vorgezogen wird, ist Teil des
\underline{Grundkörpers} und nicht des Erweiterungskörpers (siehe Abschnitt \ref{subsec:apendix-erweiterungskoerper})

\underline{3. Hinweis}: $n$ ist auch Teil des Grundkörpers und wird deshalb auch $\mod 2$ gerechnet

\section{Erweiterungskörper}

\subsection{Idee}

Erweitern des Grundkörpers (z.B. $2$) mit Exponent (z.B. $4$) $\rightarrow$
$GF(2^4)$

Irreduzibles Polynom ist die Primzahl des Erweiterungskörpers z.B. in $GF(2^4)$:\\
$\displaystyle{
    p(x) = x^4 + x + 1
}$

Irreduzibles Polynom: $ggT(p(x), b(x)) = 1$

größter gemeinsamer Teiler mit einem beliebigen Polynom $b(x)$ ist 1

d.h. $p(x)$ kann nicht in Linearfaktoren zerlegt werden

für irreduzible Polynome gilt:\\
- ist durch kein Polynom ohne Rest teilbar\\
- hat keine Nullstellen

\textbf{aber:} Nullstellen sind wichtig für Nutzung des RS-Codes, daher \frqq Erfindung\flqq{} des
Elements $\alpha$, welches Nullstelle von $p(x)$ ist

$\displaystyle{
    p(\alpha) = 0
}$

am Beispiel:\\
$\displaystyle{
    p(\alpha) = \alpha^4 + \alpha + 1 = 0
}$

Analogie: \frqq Erfindung\flqq{} von $j$, sodass gilt:

$\displaystyle{
    j^2 + 1 = 0
}$

primitives Polynom: Nullstelle ($\alpha$) des primitiven Polynoms erzeugt alle Elemente (außer 0) des Erweiterungskörpers

primitives Element: Nullstelle $\alpha$ des primitiven Polynoms

\subsection{Eigenschaften von Erweiterungskörpern}

Ordnung des primitiven Elements: $2^m - 1$ im $GF(2^m)$

Erzeugung der Elemente über Potenzieren des primitiven Elements $\alpha$

zum Körper $GF(2^m)$ gehören $2^m$ Elemente ($2^m - 1$ dieser wird durch Potenzieren von $\alpha$ erzeugt)

Elemente der Erweiterungskörper sind Polynome

\textbf{Darstellung}

Erzeugung von bspw. $\alpha^3$ in $GF(2^4)$ mit irreduziblem Polynom $p(x) = x^4 + x + 1$:

$\displaystyle{
    \alpha^3 = 1 \cdot \alpha^3 + 0 \cdot \alpha^2 + 0 \cdot \alpha^1 + 0 \cdot \alpha^0
}$

dazugehörige Binärdarstellung:

$1000$

\subsection{Kürzere Codes}

\textbf{Verkürzung}

Streichen von Informationswortstellen und Codewortstellen

Distanz und damit Fehlerkorrigierbarkeit bleibt gleich

Code ist nicht mehr zyklisch

Bsp.: Verkürzung eines $C(6, 2, 5)$ um 1 auf $C(5, 1, 5)$

Äquivalent zu Einfügen von Nullen an den hohen Potenzen des Informationswortes

\textbf{Punktierung}

Streichen bestimmter Stellen aus dem Codewortbitstrom

Mindestabstand bleibt im Allgemeinen nicht erhalten

\section{BCH-Codes}

\subsection{Idee}

Wunsch: reelle Koeffizienten (reell im $GF(2^m)$ = binär)

Für Erweiterungskörper war $\alpha$ die Nullstelle des primitiven Polynoms

ABER: d'Alembert: Polynom vom Grad $m$ hat $m$ Nullstellen

Wo sind die restlichen Nullstellen der primitiven Polynome höheren Grades?

$\hookrightarrow$ wenn $\alpha$ Nullstelle von $p(x)$ ist, dann sind auch $\alpha^2, \alpha^{2^2}, \alpha^{2^3}, ..., \alpha^{2^{m-1}}$
Nullstellen ($\rightarrow$ konjugiert komplexe Nullstellen)

Ein Polynom aus konjugiert komplexen Nullstellen hat reelle (in unserem Fall binäre) Koeffizienten

im Allgemeinen sind $\alpha^{j \cdot 2^i \mod 2^m - 1}$ \;\;\;\; ($i = 0, ..., m-1$) konjugiert komplexe Nullstellen

$\hookrightarrow$ Kreisteilungsklassen

\subsection{Kreisteilungsklassen}

Sind nur vom Erweiterungskörper abhängig, nicht vom primitiven Polynom

Eine Kreisteilungsklasse enthält die Exponenten der konjugiert komplexen Nullstellen des
primitiven Polynoms

$K_j = j \cdot 2^i \mod 2^m - 1$ \;\;\;\; ($i = 0, ..., m-1$)

Bsp.: $GF(2^3)$

$K_0 = \{0\}$\\
$K_1 = \{ 1, 2, 4 \}$\\
$K_2 = \{ 2, 4, 1 \} = K_1$\\
$K_3 = \{ 3, 6, 5 \}$

D.h. um ein (Generator)polynom aufzustellen, welches reelle (=binäre) Koeffizienten hat,
braucht man alle konjugiert komplexen Nullstellen

\subsubsection{Generatorpolynom}

Konstruktion eines Codes über Vorgabe der Korrekturfähigkeit $e$ und damit $d$

RS-Codes: Generatorpolynom vom Grad $d-1$ (siehe Abschnitt \ref{subsubsec:gen-poly})

$\hookrightarrow$ d.h. $d-1$ zusammenhängende Nullstellen

Zusammenhängende Nullstellen bei BCH-Codes nur möglich, wenn alle Elemente der entsprechenden
Kreisteilungsklassen im Generatorpolynom vorhanden sind

\section{Faltungscodes}

Filterung der Eingangssequenz mit FIR-Filter

Beschreibung durch $C(n, k, [z])$

$n$: Anzahl Ausgänge\\
$k$: Anzahl Eingänge\\
$z$: Anzahl an Speicherzellen

Generatormatrix gibt Verhalten des Coders vollständig an

$\displaystyle{
    G = \begin{pmatrix}
        142\\
        345\\
        711
    \end{pmatrix}_O
}$ \;\;\;\;\;\; Zahlen in Oktaldarstellung

$\displaystyle{
    \rightarrow G = \begin{pmatrix}
        001\,100\,010\\
        011\,100\,101\\
        111\,001\,001
    \end{pmatrix}
}$

\textbf{Freie Distanz}

Gewicht (Anzahl an Einsen) der Ausgangssequenz um vom Zustand \frqq 0\flqq{} wieder zurück in den Zustand \frqq 0\flqq{}
zu kommen

\subsection{Codierung}

\subsubsection{Ein Ausgang}

Faltungscoder ohne Redundanz

Normaler FIR-Filter mit binären Koeffizienten, Generatorsequenz ist Impulsantort

Delay: $z^{-1} = D$

Impulsantort: $\vec{g}$

Inhalt der Speicherzellen: $\vec{d}$

Ausgang: $\displaystyle{
    a = \vec{g}^T \cdot
    \begin{pmatrix}
        i\\
        \vec{d}
    \end{pmatrix}
}$

\textbf{Beispiel}

\includegraphics[width=7cm]{img/faltungscoder1in1out.PNG}

Impulsantort: $\displaystyle{
    g = 1 + 1D + 0D^2 + D^3 = 1 + D + D^3
}$

Generatorsequenz: $\displaystyle{\vec{g}^T =
    \begin{pmatrix}
        1 & 1 & 0 & 1
    \end{pmatrix}
}$

Von links nach rechts steht \textcolor{red}{$ (1 0 0) $} in den Speicherzellen und es wird $i = 1$ hineingeschrieben

$\displaystyle{
    \hookrightarrow \vec{d} = \begin{pmatrix}
        1\\
        0\\
        0
    \end{pmatrix}
}$

$\displaystyle{
    a = \begin{pmatrix}
        1 & 1 & 0 & 1
    \end{pmatrix}
    \cdot
    \begin{pmatrix}
        1\\
        1\\
        0\\
        0
    \end{pmatrix}
    = 0    
}$

\subsubsection{Mehrere Ausgänge}

Filter mit mehreren Ausgängen

$n$ Impulsantorten des Filters geben $n$ verschiedene Ausgänge

Generatorsequenz wird zu Generatormatrix mit $n$ Generatorsequenzen

$\displaystyle{
    \vec{a} = G^T \cdot
    \begin{pmatrix}
        i\\
        \vec{d}
    \end{pmatrix} =
    \begin{pmatrix}
        a^{(1)}\\
        a^{(2)}
    \end{pmatrix}
}$\;\;\;\;\;\;\;\;\;\;\;\;\;
$\displaystyle{
    G^T =
    \begin{pmatrix}
        \vec{g_1}^T\\
        \vec{g_2}^T
    \end{pmatrix}
}$

Ausgangssequenz: $a = a^{(1)}_1, a^{(2)}_1, a^{(1)}_2, a^{(2)}_2, ...$

\textbf{Beispiel}

\includegraphics[width=7cm]{img/faltungscoder1in2out.PNG}

Im Speicher steht wieder von links nach rechts \textcolor{red}{$ (1 0 0) $} und es wird $i = 1$ hineingeschrieben

$\displaystyle{
    \vec{a} =
    \begin{pmatrix}
        1 & 1 & 0 & 1\\
        1 & 1 & 1 & 1
    \end{pmatrix}
    \cdot
    \begin{pmatrix}
        1\\
        1\\
        0\\
        0
    \end{pmatrix} =
    \begin{pmatrix}
        0\\
        0
    \end{pmatrix}
}$

\subsubsection{Mehrere Eingänge}

Mehrere Eingänge ($i^{(1)}, i^{(2)}$) um Coderate anzupassen

für jeden Ausgang gibt es jeweils $n$ Generatorsequenzen ($g_{ij}$)

$i$: Eingang\\
$j$: Ausgang

\textbf{Beispiel}

\includegraphics[width=7cm]{img/faltungscoder2in3out.PNG}

$\displaystyle{
    \vec{g_{11}}^T = \begin{pmatrix}
        1 & 1 & 0 & 1
    \end{pmatrix}
    \;\;\;\;\;\;\;
    \vec{g_{21}}^T = \begin{pmatrix}
        0 & 0 & 0 & 0
    \end{pmatrix}
}$

$\displaystyle{
    \vec{g_{12}}^T = \begin{pmatrix}
        1 & 1 & 1 & 1
    \end{pmatrix}
    \;\;\;\;\;\;\;
    \vec{g_{22}}^T = \begin{pmatrix}
        0 & 0 & 1 & 0
    \end{pmatrix}
}$

$\displaystyle{
    \vec{g_{13}}^T = \begin{pmatrix}
        0 & 0 & 0 & 1
    \end{pmatrix}
    \;\;\;\;\;\;\;
    \vec{g_{23}}^T = \begin{pmatrix}
        0 & 1 & 1 & 0
    \end{pmatrix}
}$
\\

Zusamenfassung in $G$

\begin{equation*}
    G = 
    \begin{tikzpicture}[baseline={-0.5ex},mymatrixenv]
        \matrix [mymatrix,inner sep=4pt] (m)  
        {
            1 & 0 & 1 & 0 & 0 & 0 & 1 & 0\\
            1 & 0 & 1 & 0 & 1 & 1 & 1 & 0\\
            0 & 0 & 0 & 1 & 0 & 1 & 1 & 0\\
        };
    
        % Braces
        \mymatrixbracebottom{2}{1}{$D^0$}
        \mymatrixbracebottom{4}{3}{$D^1$}
        \mymatrixbracebottom{6}{5}{$D^2$}
        \mymatrixbracebottom{8}{7}{$D^3$}
        \mymatrixbraceleft{1}{1}{Ausgang 1}
        \mymatrixbraceleft{2}{2}{Ausgang 2}
        \mymatrixbraceleft{3}{3}{Ausgang 3}
        \mymatrixbracetop{1}{1}{E1}
        \mymatrixbracetop{2}{2}{E2}
        \mymatrixbracetop{3}{3}{E1}
        \mymatrixbracetop{4}{4}{E2}
        \mymatrixbracetop{5}{5}{E1}
        \mymatrixbracetop{6}{6}{E2}
        \mymatrixbracetop{7}{7}{E1}
        \mymatrixbracetop{8}{8}{E2}
    \end{tikzpicture}
\end{equation*}

\subsection{Modifikationen}

\textbf{Punktierung}

Streichen bestimmmter Stellen aus dem Codewort

Punktierung über Punktierungsmatrix gegeben

Bsp.: für einen Coder mit 2 Ausgängen\\
\begin{equation*}
    G = 
    \begin{tikzpicture}[baseline={-0.5ex},mymatrixenv]
        \matrix [mymatrix,inner sep=4pt] (m)  
        {
            1 & 0\\
            1 & 1\\
        };
        % Braces
        \mymatrixbraceleft{1}{1}{Ausgang 1}
        \mymatrixbraceleft{2}{2}{Ausgang 2}
    \end{tikzpicture}
\end{equation*}

Ausgangssequenz $a$ punktieren durch bitweise AND mit der Matrixssequenz $p = (1\;1\;0\;1)$:\\

$a = \;\;1001\;0100\;1011\;1001\;0100$\\
$p = \;\;1101\;1101\;1101\;1101\;1101$

$a^* =  10x1\;01x0\;10x1\;10x1;01x0$

$x$: gestrichene Stelle, wird nicht übertragen (wird im Trellis immer als Fehler gewertet)

\textbf{Terminierung}

Endzustand des Coders bekannt, durch Anhängen von $z$ Nullen

$\hookrightarrow$ damit ist garantiert, dass im Coder am Ende nur Nullen stehen

\textbf{Tail-Biting}

Endzustand nach Codierung eines Informationswortes wird als Anfangszustand für eine erneute Codierung
verwendet (bei FIR-Filtern ist eine erste Codierung nicht notwendig, da der Endzustand nur von den letzten
$z$ Informationsbits abhängt)

$\hookrightarrow$ damit ist der Anfangszustand identisch mit Endzustand

$\hookrightarrow$ der Decoder decodiert und kennt mit großer Sicherheit den Anfangszustand und damit auch den
Endzustand

\subsection{Decodierung}

Mit Viterbi-Algorithmus

Konstruktion eines Zustandsgraphen für den Coder

Aus Zustandgraph wird Trellis-Diagramm erstellt

\includegraphics[width=3cm]{img/zustandsgraph.PNG}
\;\;\;\;
\includegraphics[width=4cm]{img/trellis.PNG}

gesucht: Informationsfolge, welche in den Coder geschickt wurde

Metrik: Anzahl der übereinstimmenden Ausgangsbits

Pfad mit höchster Übereinstimmung wird genommen

\underline{Außer}:
\begin{itemize}
    \item bei Terminierung: Endzustand 0
    \item bei Tail-Biting: Endzustand = Anfangszustand
\end{itemize}
