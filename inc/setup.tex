\documentclass[twocolumn]{scrartcl}

\usepackage[utf8]{inputenc}
\usepackage[ngerman]{babel}
\usepackage[T1]{fontenc}
\usepackage{graphicx}
\usepackage{listings}
\usepackage{xcolor}
\usepackage[german=guillemets]{csquotes}		%% Zitieren
\usepackage{amsmath}							%% Mathe
\setcounter{MaxMatrixCols}{15}					%% pmatrix mehr als 10 Spalten
\usepackage{mathtools}
\DeclarePairedDelimiter\abs{\lvert}{\rvert}
\usepackage[hidelinks]{hyperref}				%% Ausblenden der Boxen um Referenzen
\usepackage{lmodern}							%% Schriftart
%%\usepackage[headsepline]{scrlayer-scrpage}		%% Individuelle Kopf- und Fußzeile
\usepackage[absolute]{textpos}
\usepackage{pdfpages}
\usepackage[a4paper, left=1.5cm, right=1.5cm, top=3cm, bottom=3cm]{geometry} %% Aändern der Seitenränder
\usepackage{trfsigns}
\usepackage{polynom}
\usepackage{nicematrix}
\usepackage[customcolors]{hf-tikz}
\usetikzlibrary{patterns}
\usetikzlibrary{matrix,decorations.pathreplacing}

%% TIKZ
\pgfkeys{tikz/mymatrixenv/.style={decoration={brace},every left delimiter/.style={xshift=8pt},every right delimiter/.style={xshift=-8pt}}}
\pgfkeys{tikz/mymatrix/.style={matrix of math nodes,nodes in empty cells,left delimiter={(},right delimiter={)},inner sep=1pt,outer sep=1.5pt,column sep=2pt,row sep=2pt,nodes={minimum width=20pt,minimum height=10pt,anchor=center,inner sep=0pt,outer sep=0pt}}}
\pgfkeys{tikz/mymatrixbrace/.style={decorate,thick}}

%% matrix spalten
\newcommand*\mymatrixbraceright[4][m]{
    \draw[mymatrixbrace] (#1.west|-#1-#3-1.south west) -- node[left=2pt] {#4} (#1.west|-#1-#2-1.north west);
}
\newcommand*\mymatrixbraceleft[4][m]{
    \draw[mymatrixbrace] (#1.east|-#1-#2-1.north east) -- node[right=2pt] {#4} (#1.east|-#1-#3-1.south east);
}
\newcommand*\mymatrixbracetop[4][m]{
    \draw[mymatrixbrace] (#1.north-|#1-1-#2.north west) -- node[above=2pt] {#4} (#1.north-|#1-1-#3.north east);
}
\newcommand*\mymatrixbracebottom[4][m]{
    \draw[mymatrixbrace] (#1.south-|#1-1-#2.north east) -- node[below=2pt] {#4} (#1.south-|#1-1-#3.north west);
}


\KOMAoptions
{
	fontsize=9pt,								%% Schriftgröße
	parskip=half,								%% Absatz haben Einzug(off) oder Abstand(full, half)
	headings=normal								%% Größe der Überschriften
}


%% Codeumgebung settings
% color
\definecolor{codegray}{rgb}{0.9, 0.9, 0.9}

%%Listing setup
\lstset{
	backgroundcolor=\color{codegray},
	captionpos=b,
	breaklines=true,
}

\lstdefinestyle{customc}{
	language=C,
	breaklines=true,
	frame=L,
	basicstyle=\footnotesize\ttfamily,
	commentstyle=\itshape\color{green!60!black},
	keywordstyle=\bfseries\color{blue},
	identifierstyle=\color{purple!80!black},
	stringstyle=\color{red},
	xleftmargin=\parindent,
	showstringspaces=false,
	morekeywords={uint8_t, int8_t, uint32_t},
}



%% Quellenverzeichnis (Biblatex mit biber)
\usepackage{csquotes}
\usepackage[backend=biber, style=ieee]{biblatex}
\addbibresource{content/bib/lit.bib}



%%% Einstellungen für Kopf- und Fußzeile
%\automark[section]{section}
%\clearpairofpagestyles
%\ihead{\headmark}
%\chead{}
%\ohead{\pagemark}